\documentclass{scrartcl}
\usepackage{hyperref}
\usepackage{natbib}

\begin{document}

	\title{Content Outline for Stats II Extra Stata sessions}
	\author{Max Callaghan}
	\date{November 2015}
	
	\maketitle
	
\section{Reproducible Research and Literate Programming}
This first session should give students the tools necessary to produce research output that links data, code and narrative.
\subsection{About Reproducible Research and Literate Programming}
\begin{itemize}
  \item Explanation of the principles of Reproducible Research and Literate Programming
  \begin{itemize}
    \item ``The standard of reproducibility calls for the data and the computer code used to analyze the data be made available to others" \cite{peng}
    \item Literate programming ties together data, code and the actual research output, enhancing reproducibility
  \end{itemize}
  \item How these can help students avoid errors and unnecessary repetetive tasks
  
\end{itemize}

\subsection{Tools for Reproducible Research and Literate Programming}
\begin{itemize}
  \item Git and Github
  \item Rmarkdown
\end{itemize}
\subsubsection{Options for Creating Documents that include Stata Output}
\begin{itemize}
  \item Why is copying, taking screenshots, and pasting into word suboptimal?
  \item Option 1: Use word's -link- or -includetext- fields to include log output that can be updated automatically
  \begin{itemize}
    \item Text fields can include output from log files that automatically updates when you run your do file. However, you would have to run a script to clean these log files [Need to check if this works on school computers. Also, it may be easier to make this a stata function \& this needs to be extended to control formatting. Perhaps it's possible to write a stata function that allows for including text and document writing instructions within a do file].
    \item Word is easy to use but can be frustrating when what you want to do is very specific. Formatting can be an issue.
  \end{itemize}
  \item Option 2: Use latex to write your document. Write your output tables into .tex files and graphs into image files and link to them into your document.
  \begin{itemize}
    \item This way, if you make a change to your do file which changes your output, recompiling your pdf will automatically reflect those changes.
    \item{[Is it possible to install latex on school computers?]}
  \end{itemize}
  \item Option 3: Use \href{http://homepage.stat.uiowa.edu/~rlenth/StatWeave/}{Statweave} to create one document containing instructions for writing and formatting your document as well as instructions for stata to run. [As above, is installation on school computers an issue?]
\end{itemize}

\section{Getting Started with Stata}
This, and the following sections, should give students a clearer understanding of how stata works, and should equip them with some further strategies and tools for solving problems with stata.
\subsection{General Pointers}
\begin{itemize}
\item Where to look for help
\item How to read stata manual pages
\end{itemize}

\subsection{Directory Structure}

\begin{itemize}
\item File paths and the working directory
\item How thinking about directory structure can make things easier
\end{itemize}

\subsection{Reading Data}
\begin{itemize}
\item Reading data types not formatted for stata: xlsx, csv, etc. Overcoming issues with delimiters and unhelpfully formatted data files
\item Reading data from web sources
\end{itemize}

\subsection{Processing Data}
\begin{itemize}
\item Data types
\item Generating new variables
\end{itemize}

\section{Automating Stata}

\subsection{Macros}
\begin{itemize}
  \item Using macros to store text
  \begin{itemize}
    \item Storing and evaluating a macro
    \item Using a macro to re-use a list of variables without retyping
    \item Including a macro within a macro
  \end{itemize}
  \item Using macros to store results
  \begin{itemize}
    \item Storing estimation results
  \end{itemize}
\end{itemize}

\subsection{Loops}
\begin{itemize}
  \item Looping through sequences of numbers
  \item Looping through lists of variables
\end{itemize}

\section{Presenting Results with Stata}

\subsection{Postestimation}
\begin{itemize}
  \item Accessing and using postestimation results in e()
  \item Using -predict- to generate new variables
\end{itemize}

\subsection{The -estout- program}
\begin{itemize}
  \item Storing regression output and presenting using -esttab-
  \item Computing additional statistics to add to results tables
  \item Saving tables to various file formats for use in output
\end{itemize}

\subsection{Producing Graphs}
\begin{itemize}
  \item Different types of graphs using stata
  \item Combining graphs
  \item Customising the appearance of graphs in stata
  \item Saving and presenting graphs in your output
\end{itemize}



\section*{Some Further Resources}
\begin{itemize}
  \item \url{http://data.princeton.edu/stata/}
  \item \url{https://github.com/HertieDataScience/SyllabusAndLectures}
\end{itemize}

\bibliography{stata.bib}
\bibliographystyle{plain}

\end{document}
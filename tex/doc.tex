\documentclass{beamer}

% *****************************************************************
% Estout related things
% *****************************************************************
\newcommand{\sym}[1]{\rlap{#1}}% Thanks to David Carlisle

\let\estinput=\input% define a new input command so that we can still flatten the document

\newcommand{\estwide}[3]{
		\vspace{.75ex}{
			\begin{tabular*}
			{\textwidth}{@{\hskip\tabcolsep\extracolsep\fill}l*{#2}{#3}}
			\toprule
			\estinput{#1}
			\bottomrule
			\addlinespace[.75ex]
			\end{tabular*}
			}
		}	

\newcommand{\estauto}[3]{\documentclass{beamer}
		\vspace{.75ex}{
			\begin{tabular}{l*{#2}{#3}}
			\toprule
			\estinput{#1}
			\bottomrule
			\addlinespace[.75ex]
			\end{tabular}
			}
		}

% Allow line breaks with \\ in specialcells
	\newcommand{\specialcell}[2][c]{%
	\begin{tabular}[#1]{@{}c@{}}#2\end{tabular}}

% *****************************************************************
% Custom subcaptions
% *****************************************************************
% Note/Source/Text after Tables
\newcommand{\figtext}[1]{
	\vspace{-1.9ex}
	\captionsetup{justification=justified,font=footnotesize}
	\caption*{\hspace{6pt}\hangindent=1.5em #1}
	}
\newcommand{\fignote}[1]{\figtext{\emph{Note:~}~#1}}

\newcommand{\figsource}[1]{\figtext{\emph{Source:~}~#1}}

% Add significance note with \starnote
\newcommand{\starnote}{\figtext{* p < 0.1, ** p < 0.05, *** p < 0.01. Standard errors in parentheses.}}

% *****************************************************************
% siunitx
% *****************************************************************
\usepackage{siunitx} % centering in tables
	\sisetup{
		detect-mode,
		tight-spacing		= true,
		group-digits		= false ,
		input-signs		= ,
		input-symbols		= ( ) [ ] - + *,
		input-open-uncertainty	= ,
		input-close-uncertainty	= ,
		table-align-text-post	= false
        }


\begin{document}
  \begin{frame}
    \frametitle{this is the first slide}
    Text on the first slide
  \end{frame}
  
  \begin{frame}
    \frametitle{this is the second slide}
    \framesubtitle{more information}
    
    
{
\def\onepc{$^{\ast\ast}$} \def\fivepc{$^{\ast}$}
\def\tenpc{$^{\dag}$}
\def\legend{\multicolumn{4}{l}{\footnotesize{Significance levels
:\hspace{1em} $\dag$ : 10\% \hspace{1em}
$\ast$ : 5\% \hspace{1em} $\ast\ast$ : 1\% \normalsize}}}
\begin{table}[htbp]\centering
 \caption{Estimation results : regress
\label{stab}}
\begin{tabular}{l r @{} l c }\hline\hline 
\multicolumn{1}{c}
{\textbf{Variable}}
 & \multicolumn{2}{c}{\textbf{Coefficient}}  & \textbf{(Std. Err.)} \\ \hline
income\_est  &  11.681&\onepc  & (0.222)\\
Intercept  &  16142.067&\onepc  & (139.382)\\
\hline\multicolumn{4}{c}{}\\
\hline N & \multicolumn{3}{c}{7133}\\
R$^{2}$ & \multicolumn{3}{c}{0.28}\\
F $ _{(1,7131)}$ & \multicolumn{3}{c}{2766.551}\\
\hline\legend
\end{tabular}
\end{table}
}


  \end{frame}
  
  \begin{frame}
    \frametitle{this is the second slide}
    \framesubtitle{more information}
    
    
{
\begin{table}[htbp]\centering
 \caption{Estimation results : xtreg
\label{tabresult xtreg}}
\begin{tabular}{l c c }\hline\hline 
\multicolumn{1}{c}
{\textbf{Variable}}
 & {\textbf{Coefficient}}  & \textbf{(Std. Err.)} \\ \hline
income\_est  &  20.111  & (0.237)\\
Intercept  &  11040.143  & (145.751)\\
\hline\end{tabular}
\end{table}
}


  \end{frame}
  
  \begin{frame}
    \frametitle{this is the second slide}
    \framesubtitle{more information}
    
    {
\def\sym#1{\ifmmode^{#1}\else\(^{#1}\)\fi}
\begin{tabular}{l*{2}{c}}
\hline\hline
            &\multicolumn{1}{c}{(1)}&\multicolumn{1}{c}{(2)}\\
            &\multicolumn{1}{c}{energy\_consumption}&\multicolumn{1}{c}{energy\_consumption}\\
\hline
income\_est  &       11.68\sym{***}&       20.11\sym{***}\\
            &     (52.60)         &     (84.86)         \\
[1em]
\_cons      &     16142.1\sym{***}&     11040.1\sym{***}\\
            &    (115.81)         &     (75.75)         \\
\hline
\(N\)       &        7133         &        7133         \\
\hline\hline
\multicolumn{3}{l}{\footnotesize \textit{t} statistics in parentheses}\\
\multicolumn{3}{l}{\footnotesize \sym{*} \(p<0.05\), \sym{**} \(p<0.01\), \sym{***} \(p<0.001\)}\\
\end{tabular}
}

  \end{frame}
  

\end{document}